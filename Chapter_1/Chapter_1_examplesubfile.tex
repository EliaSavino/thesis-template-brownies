\documentclass[main_brownies.tex]{subfiles}
\begin{document}
\graphicspath{ {Figures_Schemes/} } % define graphics path
\newrefsection[Chapter_1/References_Chapter_1] %% begin new reference section with the .bib file for this chapter

%% an if-else statement for inserting an empty page when necessary
	%% to make sure chapter image page is to the left of the chapter title page
\checkoddpage
\ifoddpage
	\newpage\thispagestyle{empty}
	\mbox{}
	\newpage
	\includepdf[pages={-}]{chapterimagepage_1.pdf}
\else
	\includepdf[pages={-}]{chapterimagepage_1.pdf}
\fi

	%%% try to get the chapterimagepage directly after TOC (so no 2 blank pages)

%%%TITLE AND ABSTRACT%%%

\chapter[Example Introduction Chapter]{Example of an Introduction Chapter}\blfootnote{Footnote used for if chapter has been published (or is in the process of).} %% the [short title] will end up in the TOC as well

\begin{abstract}
	\Blindtext[1][3]
\end{abstract}

%%%END OF TITLE, AUTHORS AND ABSTRACT%%%

%%%MAIN TEXT%%%%

%===================== SECTION 1 INTRODUCTION %%%%%%%%%%%
\section{Introduction} \label{section_1_introduction}
\Blindtext[1][1]

\begin{scheme}[htb]
	\centering
%	\includegraphics[width=.9\textwidth]{/} % need example iamge
	\caption{Caption of Scheme.}
	\label{sch:Scheme_1}
\end{scheme}

\FloatBarrier % makes sure all floats are inserted before the following paragraph

As you can see in Scheme~\ref{sch:Scheme_1}, blablabla.

\subsection{Subsection of Introduction} \label{section_1_1}
\Blindtext[1][1]

\section{Section 2}  \label{section_2}

\begin{figure}[htb]
	\centering
	\includegraphics[width=.9\textwidth]{Figure_1/Figure_1} %% example image
	\caption{Caption of Figure.}
	\label{fgr:Figure_1}
\end{figure}

Examples of formatting and such:
\begin{itemize}
	\item 5 wt\%
	\item 80 \degree C or math-mode: $\degree$C
	\item this is a bullet $\bullet$ 
	\item You can do textsubscript\textsubscript{like so} and textsuperscript\textsuperscript{like so} that follow main font style
	\item Unbreakable hy\=/phe\=/na\=/tion
	\item For certain fonts it is possible to use sans-serif mathfont: $\mathsf{\lambda_{exc}}$ 785 nm. Otherwise regular mathfont ($italic$) can be used: at $\lambda_{exc}$ 785 nm.
	\item open \vs{} closed %% defined command in pre-amble of main.tex % when using: type \vs{} to avoid problems with whitespace after the command
	\item 966 \cm{} and 1170 \cm{}
	\item mu not italics and follows font style: \textmu
	\item $\oldsqrt{old style}$ and $\sqrt{new style}$, see main for details
	\item How to cite.\cite{Darwin1859} 
	\item How to cite a section: Section~\ref{section_1_introduction}.
	\item The following is used to separate the next paragraph from the previous with a whitespace (e.g., \textsc{smallskip, medskip, bigskip}) and without indent (wrap text in \textsc{noindent}):
	%% to insert a line break % https://www.overleaf.com/learn/latex/Line_breaks_and_blank_spaces
\end{itemize}

\noindent\dotfill

\Blindtext[1][1]
\medskip
\noindent{\blindtext[1]}

\noindent\dotfill

\section{Section 3}
\subsection{Section 3.1}
\subsection{Section 3.2}


%%%% END OF MAIN TEXT %%%%

%%%% START OF CONCLUSIONS %%%%

\section{Conclusions}
\Blindtext[1][2]

\section*{Acknowledgements}
\addcontentsline{toc}{section}{Acknowledgements} % to add section* to TOC
\markright{\MakeUppercase{Acknowledgements}} % to add section* to fancyheader
\vspace*{-.8\baselineskip} % to remove vertical space between header and subsequent text. 
	%% Use as last resort (together with other manual adjustments) to make text fit nicely
\textsc{vertical space between header and text below can be changed in this way.}

Someone is gratefully acknowledged.

\section*{Thesis Outline}
\addcontentsline{toc}{section}{Thesis Outline} % to add section* to TOC
\markright{\MakeUppercase{Thesis Outline}} % to add section* to fancyheader

Put this at the end of the Introduction Chapter (after Conclusions and, if applicable, Acknowledgements).

This thesis is about ...

\textbf{Chapter~2} concerns ...

In \textbf{Chapter~3}, we investigate ...

Etcetera, etcetera

%%% END OF CHAPTER 1 %%%

\FloatBarrier %% to put next paragraph after latest figure
%\clearpage % use this to print all floats before this point
%\newpage

%%%==================== REFERENCES
{\raggedright
\printbibliography} %% use to loosen restrictions on justification for References section. 
	% If \raggedright is not used, there will be many warnings for overfull hboxes

%%%END OF REFERENCES%%%

%===
\end{document}