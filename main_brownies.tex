%\documentclass[print]{dissertation_brownies} %% use option print to add cropmarks and other things (see .cls)
\documentclass{dissertation_brownies} %% use option print to add cropmarks and other things (see .cls)

%% these packages are used in this template 
	%% but can be removed when writing your thesis
\usepackage{lipsum} %% to use lorem ipsum for example texts
\usepackage{blindtext} %% to use blindtext for more document-like fake text, as well as lists, etc.
\usepackage{showexpl} %% need for Instructions chapter; an environment to show examples
\usepackage{fancyvrb} %% for verbatim environment that allows line breaks
\usepackage{fvextra} %% extras for verbatim environment
%%

%%%%%%%%%%%%%%%%%%%%%%%%%%%%%%%%%%%%%%%%%

	%%% START BIBLIOGRAPHY SOURCES %%%

\addbibresource{Chapter_1/References_Chapter_1.bib} %% a separate .bib file for each chapter

\addbibresource{Chapter_2/References_Chapter_2.bib} %% a separate .bib file for each chapter

\addbibresource{Instructions/References_Instructions.bib} %% a .bib file for the Instructions Chapter

\addbibresource{Publications/My_publications.bib} %% for final page Publications

	%%% END BIBLIOGRAPHY SOURCES %%%

%%%%%%%%%%%%%%%%%%%%%%%%%%%%%%%%%%%%%%%%%

	%%% START SUBFILES %%%
\usepackage{subfiles} % Best loaded last in the preamble
%% could also use the package standalone instead of subfiles -- importantly, with standalone, all subfiles need to have their own preambles, which means you'd need to be really sure that the formatting and everything of all chapters are the same.

%% can comment out \subfile{"chapter"} to not include it -- for if you want to see how just one chapter (or a few) look(s)
%% https://tex.stackexchange.com/questions/513490/chapter-bibliography-using-subfiles

	%%% END SUBFILES %%%

%%%%%%%%%%%%%%%%%%%%%%%%%%%%%%%%%%%%%%%%%

	%%% START commands and such to fix typesetting issues %%%

%%%%%%%%%%% to fix overfull \hbox in TOC due to 3-digit pagenumber:
%% https://tex.stackexchange.com/questions/49887/overfull-hbox-warning-for-toc-entries-when-using-memoir-documentclass

%\makeatletter
%\renewcommand{\@pnumwidth}{2em}  % width for pagenumber; useful in case of 3-digit pagenumbers; default is 1.55em
%\renewcommand{\@tocrmarg}{4em} % width between righthandmargin and start of title text; default is 2.55em 
%\makeatother

%%%%%%%%%%% 

\raggedbottom %% remove unwanted spaces between paragraphs: recommended!

	%%% END commands and such to fix typesetting issues %%%

%%%%%%%%%%%%%%%%%%%%%%%%%%%%%%%%%%%%%%%%%

	%%% START possible (re)definitions %%%

%%%%%%
%% The regular square root symbol does not have the vertical bar at the end -- this changes the sqrt symbol:
\let\oldsqrt\sqrt
\def\sqrt{\mathpalette\DHLhksqrt}
\def\DHLhksqrt#1#2{%
		\setbox0=\hbox{$#1\oldsqrt{#2\,}$}\dimen0=\ht0
		\advance\dimen0-0.2\ht0
		\setbox2=\hbox{\vrule height\ht0 depth -\dimen0}%
		{\box0\lower0.4pt\box2}}
%%%%%%

\hyphenation{spi-ro-py-rans} %% allow hyphenation only at the assigned locations
\hyphenation{nohyphenation} %% if no "-" are inserted in the {"word"}, hyphenation will not be allowed
\hyphenation{hyphenation}

\newcommand{\vs}{\emph{vs}} %% define command vs (versus) in italics % when using: type \vs{}
\newcommand{\cm}{\unit{cm^{-1}}} %% define command cm-1 superscript % when using: type \cm{}

\renewcommand{\thefootnote}{\fnsymbol{footnote}} %% use \fnsymbol for the footnote symbols

%%%%%%
%% make command to enable \chapter to start on the left page:
%% used for final page Publications to make it appear as last left page
\makeatletter
\newcommand{\enableopenany}{\@openrightfalse} 
\newcommand{\disableopenany}{\@openrighttrue} 
\makeatother
%%%%%%

%%%%%% Possible redefinition of \chapter that changes the default pagestyle of its titlepage:
%% Change pagestyle of \chapter titlepage from default plain to self-defined chaptertitlepage (see .cls)
%% Will change it for both \chapter and \tableofcontents
%% Do not use in combination with the redefinitions of ps@plain (see below) that are used to distinguish between titlepages of \tableofcontents and \chapter

%\makeatletter
%\renewcommand\chapter{
	%	\if@openright
	%		\cleardoublepage
	%	\else
	%		\clearpage
	%	\fi
	%	\thispagestyle{chaptertitlepage}%% original style: plain
	%	\global\@topnum\z@
	%	\@afterindentfalse
	%	\secdef\@chapter\@schapter}
%\makeatother
%%%%%%

%%%%%% 
%% Let pagestyle plain (used for titlepages of chapters and TOC) be the self-defined chaptertitlepage (see .cls)
\makeatletter
\let\ps@plain\ps@chaptertitlepage
\makeatother
%%%%%%

	%%% END possible (re)definitions %%%

%%%%%%%%%%%%%%%%%%%%%%%%%%%%%%%%%%%%%%%%%

	%%% DEFINE THE COLOURS USED %%%
	
\definecolor{mycyan}{cmyk}{0.85,0.05,0.15,0.25}
\definecolor{mygreen}{cmyk}{0.93,0.00,1.00,0.00}
\definecolor{mypink}{cmyk}{0.02,0.80,0.45,0.05}
\definecolor{mypurple}{cmyk}{0.50,0.85,0.00,0.00}
\definecolor{myblue}{cmyk}{0.85,0.72,0.06,0.0}
\definecolor{mygold}{cmyk}{0.02,0.13,0.91,0.00}
\definecolor{myorange}{cmyk}{0.00,0.54,0.95,0.00}
\definecolor{myskyblue}{cmyk}{0.67,0.04,0.00,0.00}
%\definecolor{myblack}{cmyk}{0.00,0.00,0.00,1.00} %% defined myblack in .cls already

	%%% END COLOURS %%%

%%%%%%%%%%%%%%%%%%%%%%%%%%%%%%%%%%%%%%%%%

%%%%%%%%%%%%%%%%%% START DOCUMENT %%%%%%%%%%%%%%%%%%%%%%%

\begin{document}
%% Specify the title and author of the thesis. This information will be used on the title page (in title/title.tex) and in the metadata of the final PDF.
\title[Subtitle]{Title}
\author{First Second}{Last}

%%%%%%%%%%%%%%%%%%%%%%%%%%%%%%%%%%%%%%%%%

	%%% START frontmatter: titlepage, table of contents %%%

\frontmatter %% sets pages numbered to lowercase roman; makes chapters not numbered
\subfileinclude{Titlepage/titlepage}

\dedication{To all the wonderful Brownies using this template}

%% Titlepage of \tableofcontents uses the pagestyle plain by default! The same goes for chapter titlepages.
	%% Let pagestyle plain be the self-defined pagestyle frontmatterTOCfirstpage (see .cls)
	%% This will only make the titlepage of the TOC in the style of frontmatterTOCfirstpage (see .cls)
	%% The rest of the TOC will be in the style frontmatter
\makeatletter
\let\ps@plain\ps@frontmatterTOCfirstpage
\makeatother

%% Insert table of contents
\tableofcontents

%% Reset pagestyle plain to style of chaptertitlepage (see .cls)
\makeatletter
\let\ps@plain\ps@chaptertitlepage
\makeatother

	%%% END frontmatter %%%

%%%%%%%%%%%%%%%%%%%%%%%%%%%%%%%%%%%%%%%%%	

	%%% START mainmatter: chapters %%%

\mainmatter % numbers chapters and sets pagenumber to normal numbers

\thumbtrue % turn on thumb indices

\renewcommand{\thumbcolor}{mycyan} % set chaptercolor for every chapter specifically. % can just add name of colour here as well
\subfileinclude{Chapter_1/Chapter_1_examplesubfile}

\renewcommand{\thumbcolor}{mygreen} % set chaptercolor for every chapter specifically. % can just add name of colour here as well
\subfileinclude{Chapter_2/Chapter_2_examplesubfile}

\renewcommand{\thumbcolor}{myblack} % set chaptercolor for every chapter specifically. % can just add name of colour here as well
\subfileinclude{Instructions/Style and printing considerations}

\thumbfalse % turn off thumb indices

	%%% END mainmatter %%%

%%%%%%%%%%%%%%%%%%%%%%%%%%%%%%%%%%%%%%%%%

	%%% START backmatter: summaries, acknowledgments, etc. %%%

\backmatter %% pagestyle different from mainmatter. Can adjust it: see .cls file

\thumbbacktrue %% turn on thumbback indices
%% defined a new thumb command (thumbback): see .cls file; added new ifthumbback to the commands lthumb and rthumb
	%% can do either without chapter number, 
	%% or with self-defined acronym/abbreviation using \myacronym{}

\renewcommand{\thumbcolor}{myorange} %% same colour for all summaries. %% Can change it in the Summary.tex file if desired
\subfileinclude{Summary/Summary}

\renewcommand{\thumbcolor}{myskyblue}
\subfileinclude{Acknowledgements/Acknowledgements}

\thumbbackfalse % turn off thumbback indices

%%%%%%%%%%%%%%%%%%%%%%%%%%%%%%%%%%%%%%%%%

	%%% START Publications %%%
\renewcommand{\thumbcolor}{myblack}
\subfileinclude{Publications/Publications}

	%%% END Publications %%%

	%%% END backmatter %%%

%%%%%%%%%%%%%%%%%% END DOCUMENT %%%%%%%%%%%%%%%%%%%%%%%

\end{document}