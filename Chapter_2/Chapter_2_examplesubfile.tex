\documentclass[main_brownies.tex]{subfiles}
\begin{document}
\graphicspath{ {Figures/}{Schemes/} } %% graphics path for two folders: \Figures and \Schemes
\newrefsection[Chapter_2/References_Chapter_2] %% begin new reference section with the .bib file for this chapter

%% an if-else statement for inserting an empty page when necessary
	%% to make sure chapter image page is to the left of the chapter title page
\checkoddpage
\ifoddpage
	\newpage\thispagestyle{empty}
	\mbox{}
	\newpage
	\includepdf[pages={-}]{chapterimagepage_2.pdf}
\else
	\includepdf[pages={-}]{chapterimagepage_2.pdf}
\fi

%%%TITLE AND ABSTRACT%%%

\chapter[Example Experimental Chapter]{Example of an Experimental Chapter}\blfootnote{Footnote used for if chapter has been published (or is in the process of).} %% the [short title] will end up in the TOC as well

%% image on chapter title page if desired
\begin{center}
	\begin{minipage}[htb]{.75\textwidth}
		\centering
		\includegraphics[width=\textwidth]{titlepageimage/Chapter_2_titlepageimage} % need to have example image
	\end{minipage}
\end{center}
%%

\begin{abstract}
	\Blindtext[1][3]
\end{abstract}

%%%END OF TITLE, AUTHORS AND ABSTRACT%%%
%\clearpage
%\newpage
%%%MAIN TEXT%%%%
%===================== INTRODUCTION %%%%%%%%%%%
\section{Introduction}
\Blindtext[2][2]

\section[Short R\&D]{Results \& Discussion}  %% the [short title] will end up in the TOC as well
\subsection{R\&D 1}
\Blindtext[5][1]

\begin{itemize}
	\item Refer to a figure in Appendix/Supporting Information: 
	\begin{itemize}
		\item Figure~\ref{fgr:Figure_S1}
		\item Table~\ref{tbl:Table_S1}
	\end{itemize}
\end{itemize}

\subsection{R\&D 2}

\section{Conclusions}
\Blindtext[1][2]

\section*{Acknowledgements}
\addcontentsline{toc}{section}{Acknowledgements} % to add section* to TOC
\markright{\MakeUppercase{Acknowledgements}} % to add section* to fancyheader

Someone is acknowledged.

%%% END OF MAIN TEXT %%%

%\FloatBarrier % to put next paragraph after latest figure
%\clearpage % use this to print all floats before this point
%\newpage

%%%==================== REFERENCES

{\raggedright
\printbibliography} %% use to loosen restrictions on justification for References section. 
	% If \raggedright is not used, there will be many warnings for overfull hboxes 

%%%END OF REFERENCES%%%

%%% START OF EXPERIMENTAL DETAILS %%%
\section*{Experimental Details}
\addcontentsline{toc}{section}{Experimental Details}
\markright{\MakeUppercase{Experimental Details}}
\subsection*{Methods}

\subsection*{Synthetic Procedures}
\subsubsection*{Compound 1}

\clearpage
\newpage

\section*{Appendix}
\addcontentsline{toc}{section}{Appendix}
\markright{\MakeUppercase{Appendix}}
\vspace*{-\baselineskip} % to remove space between header and figure (float)

\begin{figureSI}[H] % use H to place the float exactly here
	\centering
	\includegraphics[width=.65\textwidth]{Figure_S1/Chapter_2_Figure_S1} % need example image
	\caption{Caption figure SI/Appendix.}
	\label{fgr:Figure_S1}
\end{figureSI}

\renewcommand{\arraystretch}{2.0} % adjust cell height for tables
\begin{tableSI}[ht]
	\centering
	%	\small
	\caption{Caption for Table using packages \textsc{adjustbox}, \textsc{multirow} and \textsc{makecell}. For example for the tabulation of vibrational modes and frequencies. $^a$note.}
	\begin{adjustbox}{width=1\textwidth}
		\begin{tabular}{|lV{2.5}c|cV{2.5}c|cV{2.5}c|c|}\hline%
			\multirow{2}{*}{\makecell[lc]{Vibrational\\ Mode}} & \multirow{2}{*}{\textbf{1}} & \multirow{2}{*}{\makecell[cc]{\textbf{1} with base\\ (deprotonation)}} & \multirow{2}{*}{\textbf{2}} & \multirow{2}{*}{\makecell[cc]{\textbf{2}\\ at low pH}} & \multicolumn{2}{c|}{\textbf{3}} \\\cline{6-7}
			& & & & & \makecell[cc]{Before\\ heating} & \makecell[cc]{After\\ heating} \\\hline\hline
			\makecell[lc]{1\cite{Einstein1905,Einstein1906}} & - & - & \# & \# & \# & \# \\\hline
			\makecell[lc]{2\cite{Einstein1905}} & \makecell[cc]{under solvent \\ band; \#\cite{Einstein1906}} & \makecell[cc]{under solvent \\ band} & - & \makecell[cc]{\#\\ (Calc.)\cite{Einstein1905}} & - & - \\\hline
			\makecell[lc]{3- \\ 3- \\ 3 \& 4} & \# & \# & \#$^a$ & \# & \# & \# \\\hline
			\makecell[lc]{4- \\ 4 \\ \& 5} & \# & \# & \#$^a$ & \# & \# & \# \\\hline
			\makecell[lc]{5 \\ and 6} & \# & \# & \#$^a$ & \# & \# & \# \\\hline
			\makecell[lc]{7- \\ 7} & \# & \# & \#$^a$ & \# & \# & \# \\\hline
			\makecell[lc]{8} & \# & \# & - & \# & - & - \\\hline
		\end{tabular}
	\end{adjustbox}
	\label{tbl:Table_S1}
\end{tableSI}

\FloatBarrier % to put next paragraph after latest figure

\subsection*{Computational Details}
\subsubsection*{Frequency Calculations}

\renewcommand{\arraystretch}{1.5} % adjust cell height for tables

\begin{center}
%	\centering
	\footnotesize %% can change font size (e.g., scriptsize, small, etcetera)
	\captionsetup{type=tableSI}
	\caption{Caption of a long table that can extend over multiple pages. For example for calculated frequencies.}
	\begin{longtable}{|l|l|l|}
		\hline
		Frequency (\cm{}) & Raman activity & Raman intensity \\\hline\hline
		1 & 20 & 55000\\\hline
		2 & 30 & 213\\\hline
		3 & 50 & 8555995\\\hline
		4 & 80 & 213\\\hline
		5 & 100 & 54598444.4\\\hline
		6 & 200 & 234.51\\\hline
		7 & 2 & 2315849.98\\\hline
		8 & 680 & 8759\\\hline
		9 & 35 & 123.4\\\hline
		10 & 87 & 92\\\hline
	\end{longtable}
%	\label{tbl:Table_S2}
\end{center}

\renewcommand{\arraystretch}{1.0} % set table row spacing back to default 1.0

\clearpage % 

\subsubsection*{Cartesian Coordinates}
Below are listed the Cartesian XYZ coordinates (\AA) \textsc{in a non-labelled and non-captioned table}.

\medskip

\begin{center}
	%	\centering
	\footnotesize
	\begin{tabular}{@{}llll}
		\textbf{Molecule~1} & & & \\		
		12 & & &\\
		 & & & \\
		H & -1.242909 & -2.152782 & 0.000000\\
		C & -0.695566 & -1.204756 & 0.000000\\
		C &  0.695566 & -1.204756 & 0.000000\\
		H &  1.242909 & -2.152782 & 0.000000\\
		C &  1.391133 & 0.000000 & 0.000000\\
		H &  2.485819 & 0.000000 & 0.000000\\
		C &  0.695566 & 1.204756 & 0.000000\\
		H &  1.242909 & 2.152782 & 0.000000\\
		C & -0.695566 & 1.204756 & 0.000000\\
		H & -1.242909 & 2.152782 & 0.000000\\
		C & -1.391133 & 0.000000 & 0.000000\\
		H & -2.485819 & 0.000000 & 0.000000\\
	\end{tabular}
\end{center}

%===
\end{document}